%%%%%%%%%%%%%%%%%%%%%%%%%%%%%%%%%%%%%%%%%%%%%%%%%%%%%%%%%%%%%%%%%%%%%%%
% Configuração dos elementos pré-textuais

%----------------------------------------------------------------------
% Opcionais - Dedicatória, Agradecimento e Epígrafe

\dedicatoria{Aos meus pais, cuja fé em mim me ensinou a ter fé em mim mesmo.}

\agradecimento{Sou grata aos meus pais pela vida, pois sem vocês minha existência não seria possível. Sou grata pelo carinho, apoio e amor incondicional depositados em mim todos os dias, e desta forma, posso ser quem sou hoje. Por serem meu exemplo de esforço e dedicação, por me mostrarem que eu sou capaz de trilhar o caminho que desejo e conquistar o que almejo.

Sou grata ao meu namorado, por todo o apoio, paciência, e compreensão da minha ausência em horas de estudo.

Sou grata aos meus amigos, pelo companherismo, alegrias, tristeza e principalmente pela amizade. Foi a cumplicidade do dia-a-dia de vocês e o nosso esforço como equipe que me ajudaram a chegar aonde cheguei.

Sou grata à universidade pela oferta do curso e à oportunidade de fazer parte da família nesse período de aprendizado.

Sou grata aos professores, pelo tempo e esforço em compartilhar todo o seu conhecimento, em especial ao meu orientador, pelo empenho dedicado à elaboração deste trabalho.

Sou grata a todos que direta ou indiretamente fizeram parte da minha formação.

Por fim, sou grata à Deus, pela oportunidade de estar nesta Terra e de poder conviver com essas pessoas, sempre aprendendo e evoluindo.}

\epigrafe{A música, está em tudo ao nosso redor. Tudo o que temos que fazer é ouvir.}{\citeonline{kirsten2008}}

%\epigrafe{Happiness can be found, even in the darkest of times, if one only remembers to turn on the light.}{J.K. Rowling, Harry Potter and the Prisoner of Azkaban}

%----------------------------------------------------------------------
% Resumo - 150 a 500 palavras

\textoResumo {O som não é algo que podemos ver com nossos olhos. Então, o que é som? O som é a variação da pressão do ar. Sendo assim, a forma de produzir um determinado som depende da maneira como a pressão do ar varia. Representar o som numericamente é chamado de \textit{digitalização}, ou seja, é representar uma onda sonora (áudio analógico) em código binário (áudio digital). Um processo de transformação e compressão do som, reduzindo o seu tamanho, pois armazenar dados de áudio em formato analógico consome muito espaço. Em paralelo, tornou-se possível codificar arquivos de áudio com mais informação do que apenas o nome do arquivo, utilizando os \textit{metadados}. Com os metadados é possível nomear, descrever, catalogar e indicar os direitos de autor de um arquivo de áudio digital. Desta forma, um arquivo de áudio digital é composto por metadados e som digitalizado, sendo assim, um dado musical. O dado é a representação física de um evento no tempo e espaço que não agrega fundamento, não podendo ser possível entender o que ele representa ou para que ele existe, porém, ao incluir um “significado” no dado e gerar sentido para quem o ouve e ficando claro ou não a que se refere, é gerada a informação. Uma informação musical apresenta determinadas especificidades de comportamento na sua produção, objetivação e uso. Assim, a música tem diferentes significações para cada indivíduo. A música era um meio de comunicação exclusivamente presencial e com a evolução dos inventos tecnológicos, a música ultrapassa os limites físicos da mídia, mergulhando no universo digital. Desta forma, o problema de representação e o processo de construção de sistemas de processamento e recuperação musicais agrava-se com a necessidade de desenvolvimento de sistemas com estruturas internas o mais compatível possível com as visões ou desejos dos usuários. Portanto, a relevância deste trabalho contribui, diretamente, para agregar conhecimento com o estudo sobre a recuperação da informação de dados musicais que auxiliarão no desenvolvimento futuro de soluções para busca por similaridade de dados musicais. Especificamente, este trabalho visa apresentar e comparar soluções para recuperação de informação musical. A intenção é analisar soluções que não necessariamente buscam dados musicais apenas através do casamento direto de parâmetros de entrada para a busca, como título da música, palavras-chave ou um áudio com parte da música, mas também através do casamento aproximado (ou similar) destes parâmetros.} 

\palavrasChave {Recuperação da informação musical. Busca por similaridade. Dados musicais. Áudio digital. Metadados.}

%----------------------------------------------------------------------
% Abstract

\textAbstract {Sound is something we can't see. So, what is sound? Sound is the variation of air pressure. The way to produce a certain sound depends the air pressure varies. Representing the sound numerically is called \textit{scan}, it is to represent a sound wave (analog audio) in binary code (digital audio). A process of transformation and compression of sound, reducing its size, since storing audio data in analog format consumes a lot of space. In parallel, it became possible to encode audio files with more information than just the file name, using \textit{metadata}. With metadata it is possible to name, describe, catalog and indicate the copyright of a digital audio file. In this way, a digital audio file is composed of metadata and digitized sound, thus being a musical data. The data is the physical representation of an event in time and space that does not add a foundation, and it can not be possible to understand what it represents or for what it exists, however, by including a "meaning" in the given and generating meaning for the listener and whether or not it is clear what it refers to, the information is generated. A musical information presents certain specificities of behavior in its production, objectification and use. Thus, music has different meanings for each individual. Music was a means of exclusively on-site communication and with the evolution of technological inventions, music surpasses the physical limits of the media, plunging into the digital universe. In this way, the problem of representation and the process of construction of musical processing and recovery systems is aggravated by the need to develop systems with internal structures as compatible as possible to the visions or desires of the users. Therefore, the relevance of this work contributes, directly, to aggregate knowledge with the study on the retrieval of musical data information that will aid in the future development of solutions for searching for similarity of musical data. Specifically, this work aims to present and compare solutions for music information retrieval. The intention is to analyze solutions that do not necessarily search for musical data only through direct marriage of input parameters to the search, such as song title, keywords or an audio with part of the song, but also through approximate (or similar) these parameters.}

\keywords {Retrieval of musical information. Search for similarity. Musical data. Digital audio. Metadata.}
