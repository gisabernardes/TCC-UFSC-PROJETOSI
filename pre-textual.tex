%%%%%%%%%%%%%%%%%%%%%%%%%%%%%%%%%%%%%%%%%%%%%%%%%%%%%%%%%%%%%%%%%%%%%%%
% Configuração dos elementos pré-textuais

%----------------------------------------------------------------------
% Opcionais - Dedicatória, Agradecimento e Epígrafe

\dedicatoria{Ao meu namorado, por toda paciência, compreensão, carinho e amor}

\agradecimento{agradecimento}

\epigrafe{Happiness can be found, even in the darkest of times, if one only remembers to turn on the light.}{J.K. Rowling, Harry Potter and the Prisoner of Azkaban}


%----------------------------------------------------------------------
% Resumo

\textoResumo {O som não é algo que podemos ver com nossos olhos. Então, o que é som? O som é a variação da pressão do ar. Sendo assim, a forma de produzir um determinado som depende da maneira como a pressão do ar varia. Representar o som numericamente, é chamá-lo de dado, e incluir um "significado" implícito nesse dado, é gerar informação. Uma informação musical apresenta determinadas especificidades de comportamento na sua produção, objetivação e uso. Assim, a música tem diferente significações para cada indivíduo. A música era um meio de comunicação exclusivamente presencial e com a evolução dos inventos tecnológicos a música ultrapassa os limites físicos da mídia, mergulhando no universo digital. Desta forma, o problema de representação e o processo de construção de sistemas de processamento e recuperação musicais agrava-se com a necessidade de desenvolvimento de sistemas com estruturas internas o mais compatível possível com as visões ou desejos dos usuários. Portanto, a relevância deste trabalho contribui, diretamente, para agregar conhecimento com o estudo sobre a recuperação da informação de dados musicais que auxiliarão no desenvolvimento futuro de soluções para busca por similaridade de dados musicais.}

\palavrasChave {recuperação da informação musical, busca por similaridade, dados musicais}

%----------------------------------------------------------------------
% Abstract

\textAbstract {Here is written the abstract of the document}

\keywords {key 1. key 2. ... key n.}
