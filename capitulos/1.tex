\chapter{Introdução}
Nós ouvimos uma variedade de sons à todo momento, e vivemos toda a nossa vida rodeados por eles. Sons de portas abrindo e fechando, dos passos, do ruído dos motores dos automóveis, da chuva e da música. O som não é algo que podemos ver com nossos olhos. Então, o que é som?

O som é a variação da pressão do ar. Sendo assim, a forma de produzir um determinado som depende da maneira como a pressão do ar varia. Representar o som numericamente, é chamá-lo de dado, e incluir um "significado" \ implícito nesse dado, é gerar informação.

Uma informação musical apresenta determinadas especificidades de comportamento na sua produção, objetivação e uso, pois a manifestação da música apresenta-se carregada de características próprias. Portanto, a compreensão completa da música está diretamente ligada com o reconhecimento do contexto histórico e social de sua origem, com a interpretação pessoal e individual do ouvinte, e com os aspectos sonoros que a constituem, dessa forma, a música tem diferente significações para cada indivíduo. Assim, a música é uma expressão humana construída socialmente e objetivada através de sua comunicação oral, registro sonoro ou representação gráfica.

Até o surgimento dos inventos tecnológicos, a música era um meio de comunicação exclusivamente presencial. Com o decorrer do tempo as técnicas e invenções aplicadas ao processo de gravação do som foram surgindo e se aperfeiçoando, resultando em aparelhos reprodutores e suportes cada vez mais versáteis e manipuláveis. A música se tornou um objeto de consumo universal e extremamente acessível.

Com a internet, a música ultrapassa os limites físicos da mídia, mergulhando no universo digital, a música passa a circular livremente pela rede mundial de computadores através do \textit{streaming} e a popularização dos aplicativos que oferecem mais de 30 milhões de música a seus usuários.

Desta forma, a organização da informação, que inclui a sua representação, tem a principal finalidade de possibilitar a recuperação dessa informação, além da sua guarda para a posteridade. A busca por similaridade musical está inserida dentro de um tema de estudos denominado \textit{Music Information Retrieval}. Sua produção se intensificou com a explosão do interesse em coleções em rede que contenham obras musicais na forma digital, possibilitadas pelo desenvolvimento da compressão de áudio. Os pesquisadores de MIR observam que a motivação maior para essa área de pesquisa é o grande volume de música digital disponível na Internet que quanto mais cresce menos possibilita sua recuperação eficiente visto que estão apenas disponíveis aos montes, mas sem o tratamento adequado.

A área de MIR conta com profissionais das mais diversas áreas inclusas na questão do tratamento e recuperação da informação musical, mas não apresenta uma ação interdisciplinar, o que prejudica todo o seu processo de comunicação científica, pois não há um periódico ou livro-texto fundador onde pessoas interessadas podem adquirir as bases teóricas e práticas de MIR. 

Com exceção de alguns pequenos encontros interdisciplinares, muitos pesquisadores estão apresentando seus resultados para membros das suas próprias disciplinas. A literatura de MIR é difícil de ser localizada, lida e estudada, o que dificulta construir e sustentar uma área de pesquisa respeitável e próspera.

Diante das dificuldades provenientes da escassa produção científica a respeito do tema, o problema de representação e o processo de construção de sistemas de processamento e recuperação musicais agrava-se com a necessidade de desenvolvimento de sistemas com estruturas internas o mais compatível possível com as visões ou desejos dos usuários.

Neste trabalho, buscou-se reunir dados/informações com o propósito de contribuir com futuros trabalhos que desejam desenvolver soluções para a busca por similaridade de dados musicais.

Portanto, a relevância deste trabalho contribui, diretamente, para agregar conhecimento com o estudo sobre a recuperação da informação de dados musicais que auxiliarão no desenvolvimento futuro de soluções para busca por similaridade de dados musicais. A pesquisa também tem como objetivo mostrar, de forma clara, o estado da arte sobre a recuperação da informação de dados musicais; Identificar formatos de dados musicais e como é feito o armazenamento deles, no banco de dados; Apresentar métodos e algoritmos utilizados para busca por similaridade de dados musicais; e por fim, apresentar e comparar algumas soluções existentes para busca de dados musicais.

\section{Objetivos}
\subsection{Objetivo Geral}
Este trabalho de conclusão de curso tem como objetivo geral estudar e realizar uma análise comparativa de abordagens para busca por similaridade de dados musicais.

\subsection{Objetivos Específicos}
 \begin{itemize}
   \item Estudar a fundamentação teórica sobre a recuperação da informação de dados musicais;
   \item Identificar formatos de dados musicais e como é feito o armazenamento deles, no banco de dados;
   \item Apresentar métodos e algoritmos utilizados para busca por similaridade de dados musicais;
   \item Apresentar e comparar algumas soluções existentes para busca de dados musicais.
 \end{itemize}

\section{Contribuições}
Este trabalho de conclusão de curso tem o propósito de contribuir com futuros trabalhos que desejam desenvolver soluções para busca por similaridade de dados musicais.