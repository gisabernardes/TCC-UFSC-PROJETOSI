\chapter{Introdução}
A música está presente em quase todas as esferas da vivência humana, é um elemento cultural e, dessa forma, tem seu alicerce nas relações sociais. Cada cultura possui características musicais próprias e variáveis, percepções rítmicas e instrumentais diferentes, bem como distintos entendimentos do papel social da música. A música é, antes de tudo, um elemento cultural resultado de um processo de significação social. Assim, a música é uma expressão humana construída socialmente e objetivada através de sua comunicação oral, registro sonoro ou representação gráfica.

Desta forma, a organização da informação, que inclui a sua representação, tem a principal finalidade de possibilitar a recuperação dessa informação, além da sua guarda para a posteridade. A busca por similaridade musical está inserida dentro de um tema de estudos denominado Music Information Retrieval (MIR), uma linha de pesquisa interdisciplinar com o objetivo de identificar diversas palavras e propriedades contidas em seleções musicais, armazenando estes dados em bases musicais e fornecendo mecanismos para consulta, que trata da descrição automática, entendimento, pesquisa, recuperação e organização de conteúdos musicais.

Diante das dificuldades provenientes da música em si, o problema de representação e processo de construção de sistemas de processamento e recuperação musicais agrava-se com a necessidade de desenvolvimento de sistemas com estruturas internas o mais compatíveis possível com as visões ou desejos dos usuários. Portanto, buscou-se reunir dados/informações com o propósito de contribuir com futuros trabalhos que desejam desenvolver soluções para busca por similaridade de dados musicais.

Portanto, a relevância deste trabalho contribui, diretamente, para agregar conhecimento com o estudo sobre a recuperação da informação de dados musicais que auxiliarão no desenvolvimento futuro de soluções para busca por similaridade de dados musicais. A pesquisa também tem como objetivo mostrar, de forma clara, o estado da arte sobre a recuperação da informação de dados musicais; Identificar formatos de dados musicais e como é feito o armazenamento deles, no banco de dados; Apresentar métodos e algoritmos utilizados para busca por similaridade de dados musicais; e por fim, apresentar e comparar algumas soluções existentes para busca de dados musicais.

\section{Objetivos}
\subsection{Objetivo Geral}
Este trabalho de conclusão de curso tem como objetivo geral estudar e realizar uma análise comparativa de abordagens para busca por similaridade de dados musicais.

\subsection{Objetivos Específicos}
 \begin{itemize}
   \item Estudar a fundamentação teórica sobre a recuperação da informação de dados musicais;
   \item Identificar formatos de dados musicais e como é feito o armazenamento deles, no banco de dados;
   \item Apresentar métodos e algoritmos utilizados para busca por similaridade de dados musicais;
   \item Apresentar e comparar algumas soluções existentes para busca de dados musicais.
 \end{itemize}

\section{Contribuições}
Este trabalho de conclusão de curso tem o propósito de contribuir com futuros trabalhos que desejam desenvolver soluções para busca por similaridade de dados musicais.