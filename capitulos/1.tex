\chapter{Introdução}
Nós ouvimos uma variedade de sons à todo momento e vivemos toda a nossa vida rodeados por eles. Sons de portas abrindo e fechando, dos passos, do ruído dos motores dos automóveis, da chuva e da música. O som não é algo que podemos ver com nossos olhos \cite{miletto2004}. Então, o que é som?

O som é a variação da pressão do ar \cite{muller2007}. Sendo assim, a forma de produzir um determinado som depende da maneira como a pressão do ar varia. Representar o som numericamente é chamá-lo de dado, e incluir um "significado" \ implícito nesse dado é gerar informação \cite{miletto2004}.

Uma informação musical apresenta determinadas especificidades de comportamento na sua produção, objetivação e uso, pois a manifestação da música apresenta-se carregada de características próprias. Portanto, a compreensão completa da música está diretamente ligada com o reconhecimento do contexto histórico e social de sua origem, com a interpretação pessoal e individual do ouvinte, e com os aspectos sonoros que a constituem. Dessa forma, a música tem diferente significações para cada indivíduo \cite{michels1992}. Assim, a música é uma expressão humana construída socialmente e objetivada através de sua comunicação oral, registro sonoro ou representação gráfica \cite{barros2012}.

Até o surgimento dos inventos tecnológicos, a música era um meio de comunicação exclusivamente presencial. Com o decorrer do tempo, as técnicas e invenções aplicadas ao processo de gravação do som foram surgindo e se aperfeiçoando, resultando em aparelhos reprodutores e suportes cada vez mais versáteis e manipuláveis \cite{daquino2012}. A música se tornou um objeto de consumo universal e extremamente acessível \cite{gomes2015}.

Com a Internet, a música ultrapassa os limites físicos da mídia, mergulhando no universo digital. Ela passa a circular livremente pela rede mundial de computadores através do \textit{streaming} \cite{junior&segundo2008} e a popularização de aplicativos que oferecem mais de 30 milhões de músicas a seus usuários.

Desta forma, a organização da informação, que inclui a sua representação, tem a principal finalidade de possibilitar a recuperação dessa informação, além da sua guarda para a posteridade. A busca por similaridade musical está inserida dentro de um tema de estudos denominado \textit{Music Information Retrieval} \cite{mclane1996}. Sua produção se intensificou com a explosão do interesse em coleções que possuam obras musicais na forma digital, possibilitadas pelo desenvolvimento da compressão de áudio. Os pesquisadores de MIR observam que a motivação maior para essa área de pesquisa é o grande volume de música digital disponível na Internet que, quanto mais cresce, menos possibilita sua recuperação eficiente visto que estão disponíveis em grande volume, mas sem o tratamento adequado \cite{gomes2015}.

Segundo \citeonline{santini&souza2007}, a área de MIR conta com profissionais das mais diversas áreas inclusas na questão do tratamento e recuperação da informação musical, mas não apresenta uma ação interdisciplinar, o que prejudica todo o seu processo de comunicação científica, pois não há um periódico ou livro-texto fundador onde pessoas interessadas podem adquirir as bases teóricas e práticas de MIR. 

Com exceção de alguns pequenos encontros interdisciplinares, muitos pesquisadores estão apresentando seus resultados para membros das suas próprias disciplinas. A literatura de MIR é difícil de ser localizada, lida e estudada, o que dificulta construir e sustentar uma área de pesquisa respeitável e próspera \cite{santini&souza2007}.

Diante das dificuldades provenientes da escassa produção científica a respeito do tema, o problema de representação e o processo de construção de sistemas de processamento e recuperação musicais agrava-se com a necessidade de desenvolvimento de sistemas com estruturas internas o mais compatível possível com as visões ou desejos dos usuários.

Este trabalho busca reunir dados/informações com o propósito de contribuir com futuros trabalhos que desejam desenvolver soluções para a busca por similaridade de dados musicais. A intenção é analisar soluções que não necessariamente buscam dados musicais apenas através do casamento direto de parâmetros de entrada para a busca, como título da música, palavras-chave ou um áudio como parte da música, mas também através do casamento aproximado (ou similar) destes parâmetros.

Portanto, a relevância deste trabalho contribui, diretamente, para agregar conhecimento com o estudo sobre a recuperação da informação de dados musicais que auxiliarão no desenvolvimento futuro de soluções para busca por similaridade de dados musicais. A pesquisa também tem como objetivo mostrar, de forma clara, o estado da arte sobre a recuperação da informação de dados musicais, identificar formatos de dados musicais e como é feito o armazenamento deles em bancos de dados ou repositórios digitais, apresentar métodos e algoritmos utilizados para busca por similaridade de dados musicais, e, por fim, apresentar e comparar algumas soluções existentes para busca de dados musicais.

%OBJETIVOS%
\section{Objetivos}
\subsection{Objetivo Geral}
Este trabalho de conclusão de curso tem como objetivo geral estudar e realizar uma análise comparativa de abordagens para busca por similaridade de dados musicais.

%==> Trocar a palavra "abordagens" por "soluções existentes" ? É utilizado abordgens no objetivo geral, mas utilizado soluções existentes nos especificos.

\subsection{Objetivos Específicos}
Os objetivos específicos necessários para se alcançar o objetivo geral são os seguintes:

 \begin{itemize}
   \item Estudar a fundamentação teórica sobre a recuperação da informação de dados musicais;
   \item Identificar formatos de dados musicais e como é feito o armazenamento deles em bancos de dados ou outros tipos de repositórios digitais;
   \item Apresentar métodos e algoritmos utilizados para busca por similaridade de dados musicais;
   \item Apresentar e comparar algumas soluções existentes para busca de dados musicais.
 \end{itemize}
 
%METODOLOGIA%
\section{Metodologia}

Esse estudo tem por finalidade realizar um projeto de pesquisa aplicada, uma vez que utilizará conhecimento da pesquisa básica para resolver problemas.

Para um melhor tratamento dos objetivos e melhor apreciação desta pesquisa, observou-se que ela é classificada como pesquisa qualitativa e exploratória. Detectou-se também a necessidade da pesquisa bibliográfica, uma vez que, a pesquisa bibliográfica implica em que os dados e informações necessárias para realização da pesquisa sejam obtidos a partir de abordagens já trabalhadas por outros autores através de livros, artigos, \textit{surveys}, revistas especializadas, documentos eletrônicos e enciclopédias, entre outras fontes \cite{ednalucia2005}.

Neste trabalho foi feito o uso e coleta de dados de material já publicado, constituído principalmente de teses, dissertações e artigos científicos na busca e alocação de conhecimento sobre recuperação da informação musical.

A primeira etapa consistiu em um levantamento do estado da arte sobre "recuperação da informação musical". A partir desse levantamento, para embasar a pesquisa, foi feito um estudo teórico sobre as tecnologias envolvidas no processo de recuperação da informação musical.

A segunda etapa focou na discriminação resumida de algumas soluções existentes para busca de dados musicais relacionados ao tema abordado.

A terceira etapa, consiste no detalhamento de características e critérios a serem utilizados na comparação das soluções ....,  .... e .... .
%ESCOLHER AS SOLUÇÕES A SEREM ANALISADAS E COMPARADAS

Na quarta etapa, será realizado uma análise comparativa entre as soluções escolhidas conforme os critérios estabelecidos na terceira etapa.

Na quinta etapa, o propósito que se pretende alcançar é retomar o objetivo proposto, fazer uma síntese do trabalho, citando os pontos relevantes e os resultados obtidos na análise. Além de apontar idéias para trabalhos futuros encontrados durante o desenvolvimento do trabalho e explanar o aprendizado adquirido.

%ESTRUTURA DO TRABALHO%
\section{Estrutura do Trabalho}
Este trabalho de conclusão de curso está organizado da seguinte forma: 

O Capítulo 2 apresenta o estado da arte sobre a recuperação da informação de dados musicais, a identificação de formatos de dados musicais e como é feito o armazenamento deles em bancos de dados ou repositórios digitais, e a apresentação de métodos e algoritmos utilizados para busca por similaridade de dados musicais.

O Capítulo 3 apresenta de forma resumida algumas soluções existentes para busca de dados musicais.

O Capítulo 4 apresenta o detalhamento dos critérios de comparação, a análise entre as soluções escolhidas e os resultados obtidos à partir da análise.

O Capítulo 5 apresenta os objetivos para a continuidade no desenvolvimento deste trabalho na disciplina de Projetos 2.