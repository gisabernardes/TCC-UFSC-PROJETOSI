\chapter{Considerações Finais} \label{cap:consideracoes-finais}

Este trabalho teve como objetivo analisar soluções para a busca por similaridade de dados musicais, que não necessariamente buscam dados musicais apenas através do casamento direto de parâmetros de entrada para a busca, como título da música, palavras-chave ou um áudio como parte da música, mas tambem atavés do casamento aproximado (ou similar) destes parâmetros.

Além do aprendizado adquirido, através de uma pesquisa aplicada sobre o que são dados musicais, sobre a forma como os dados são tratados e como são armazenados, para que posteriormente possamos recuperar essa informação e ouví-la no dia-a-dia, este trabalho buscou reunir informações com o propósito de contribuir com futuros trabalhos que desejam desenvolver soluções para a busca por similaridade de dados musicais.

Pela observação dos aspectos analisados na Tabela \ref{comparacaoCriterios}, as soluções comerciais focam na usabilidade, para que facilite o uso para o público-alvo, que é o usuário final, que apenas busca o uso de tais soluções para o dia-a-dia, onde o objetivo pode ser o reconhecimento de músicas através de uma parte da música ou da voz e a criação de playlists e download de músicas para uso off-line. O método mais utilizado para o reconhecimento de músicas é o Fingerprint.

Para as soluções acadêmicas, as aplicações focam na otimização e desempenho da funcionalidade do reconhecimento de músicas. O desenvolvimento de tais aplicações é voltado para os usuários pesquisadores da comunidade de MIR, com o objetivo de trazer inovação para a forma como é recuperado os dados musicais.

Levando-se em consideração esses aspectos, para trabalhos futuros, é possível aprofundar o estudo nos métodos e algoritmos utilizados para a recuperação da informação musical, além de uma comparação para verificar os métodos mais eficientes na recuperação da informação. Assim como o aprimoramento das soluões acadêmicas para a otimização e desempenho das pesquisas na comunidade do MIR. Inclusive a participação de mais mulheres no programa de mentoria WIMIR\footnote{https://wimir.wordpress.com/mentoring-program/} para o desenvolvimento de soluções para a recuperação da informação musical por meio de busca por similaridade de dados musicais.
