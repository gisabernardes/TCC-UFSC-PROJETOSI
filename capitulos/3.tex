\chapter{Soluções Existentes}
Nos últimos anos, várias plataformas digitais de \textit{streaming} têm surgido, derivado das intensas procuras pelo usuário. Com a variedade de serviços que temos ao nosso dispor, este capítulo apresenta de forma resumida algumas soluções existentes para busca de dados musicais, com base nos seus sites oficiais.

%MUSICID%
\section{MusicID}
\textit{Gracenote Inc.} foi fundada em 1998, é uma empresa que fornece metadados de música, vídeo e esportes e tecnologias de reconhecimento automático de conteúdo para empresas e serviços de entretenimento em todo o mundo.

Segundo o site da companhia\footnote{http://www.gracenote.com/music/music-recognition/}, tradução nossa:

\begin{citation}
O Gracenote MusicID\textregistered é o padrão para reconhecimento de música. Ele ajuda os fãs a desbloquear seus álbuns e faixas favoritos na nuvem e a descobrir novas músicas com seus celulares, além de permitir o monitoramento de músicas para detentores de direitos e profissionais do setor.
\end{citation}

O Gracenote MusicID\textregistered, faz o reconhecimento de músicas que são tocadas ao seu redor, combinado ao uso de \textit{fingerprints} e correspondência de texto para identificar arquivos de música digital a um banco de dados mundial de informações musicais. Uma vez reconhecidos, os arquivos são organizados por nome de faixa, nome do álbum e caminhos de pastas para garantir que as músicas e os álbuns certos sejam sempre correspondidos.

%SHAZAM%
\section{Shazam}
\textit{Shazam Entertainment Ltd} foi fundado em 2000 com a idéia de prover um serviço que pudesse conectar as pessoas à musica, permitindo a identificação da música através dos \textit{Smartphones}.

A aplicação usa o microfone do \textit{Smartphone} ou do computador para capturar uma pequena amostra de música, e então, realiza a identificação da música em um grande banco de dados com mais de 12 bilhões de músicas e, além disso, deve ter um baixo número de erros, ao mesmo tempo que tem uma alta taxa de acertos.

Segundo o site da companhia\footnote{https://www.shazam.com/pt/company}:
\begin{citation}
Shazam é uma aplicação móvel que reconhece música e conteúdos de TV à sua volta. É a melhor maneira de descobrir, explorar e compartilhar a música e os conteúdos de TV que você mais gosta. Levamos 10 anos para alcançar 1 bilhão de tags, 10 meses para chegar a 2 bilhões, 3 meses para ir de 10 a 12 bilhões... É uma aplicação fantástica, agora disponível nas lojas da Apple e Android. E estamos sempre à procura de novas maneiras de encantar os nossos usuários.
\end{citation}

Para o trecho de música capturado pela aplicação é criado uma \textit{fingerprint} ou impressão digital, tradução literal icorporada a palavra. Desta forma, é comparada com todas as outras \textit{fingerprints} derivadas das músicas no banco de dados. Se houver uma correspondência, é enviado informações da música para o usuário, como artista, álbum e título da música.

%SOUNDHOUND%
\section{SoundHound}
\textit{SoundHound Inc.} foi fundada em 2005, é uma empresa pioneira em desenvolvimento de aplicações para reconhecimento de voz, compreensão da linguagem natural, reconhecimento de som e tecnologias de busca.

Segundo o site da companhia\footnote{https://soundhound.com/about}, tradução nossa:

\begin{citation}
Acreditamos em permitir que humanos interajam com as coisas ao seu redor da mesma forma como interagimos entre nós: falando naturalmente com telefones celulares, carros, TVs, caixas de música, máquinas de café, e todas as outras partes emergentes do mundo "conectado". Nosso produto mais recente, Hound, utiliza a nossa tecnologia \textit{Speech-to-Meaning}\tm para mostrar uma experiência inovadora com os \textit{Smartphones}. Nosso produto \textit{SoundHound} aplica nossa tecnologia a música, permitindo as pessoas descobrir, explorar e compartilhar música ao seu redor, e até mesmo encontrar o nome daquela música presa em suas cabeças cantando ou cantarolando. E através da plataforma Houndify, capacitamos os desenvolvedores para fazerem parte dessa revolução \textit{speech-to-meaning}.
\end{citation}

É através da plataforma independente de Inteligência Artificial Houndify, combinado ao \textit{Automatic Speech Recognition} (ASR) e o \textit{Natural Language Understanding} (NLU) que permite a identificação das músicas de forma rápida e eficiente.

Desta forma, os dois produtos conhecidos no meio musical são:

\begin{enumerate}
    \item SoundHound Music Search & Play\footnote{https://soundhound.com/soundhound}: É um aplicativo para \textit{Smartphones}, onde é possível descobrir, pesquisar e reproduzir qualquer música com controle de voz. Também é possível pesquisar músicas cantando ou cantarolando, tornando o único app no mundo que pode dar resultados, imediatamente e com precisão, ouvindo você cantar ou cantarolar.
    \item Midomi\footnote{https://www.midomi.com/}: Possui as mesmas características que o item anterior, porém possui versão para \textit{web} e a versão \textit{mobile} é destinado aos modelos mais antigos de \textit{Smartphones}.
\end{enumerate}

%DEEZER%
\section{Deezer}
%2006
%Falta escrever%

%SOUNDCLOUD%
\section{SondCloud}
%2007
%Falta escrever%

%SPOTIFY%
\section{Spotify}
%2008
%Falta escrever%

%MUSIXMATCH%
\section{Musixmatch}
\textit{Musixmatch} foi fundado em 2010, com o objetivo de mudar a forma como as pessoas experimentam música e letras.

Segundo o site da companhia\footnote{http://about.musixmatch.com/}, tradução nossa:

\begin{citation}
Musixmatch é a maior plataforma de letras do mundo - onde você pode pesquisar, curtir e compartilhar letras de qualquer música, em qualquer lugar do mundo.
\end{citation}

A plataforma pode ser acessada através do site e por aplicativo nos \textit{Smartphones}. O Musixmatch digitaliza todas as músicas da biblioteca de música do usuário e encontra letras para todas elas e apresenta a letra conforme a música que estiver sendo tocada. 

Possui também, a capacidade para capturar uma pequena amostra de música (mesma função encontrada em aplicativos como o Shazam e o Soundhound), além de identificar a letra da música e mantê-la sincronizada enquanto a música é tocada.
