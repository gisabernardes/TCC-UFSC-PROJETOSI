\chapter{Soluções Existentes} \label{cap:solucoes}
Nos últimos anos, várias plataformas digitais de \textit{streaming} têm surgido, derivado das intensas procuras pelo usuário por música on-line. Com a variedade de serviços que temos ao nosso dispor, esta seção apresenta de forma resumida algumas soluções existentes para busca de dados musicais, de forma comercial e acadêmica, com base nos seus sites oficiais.

%%COMERCIAL%%
\section{Comercial}

%MUSICID%
\subsection{MusicID}
\textit{Gracenote Inc.} foi fundada em 1998, é uma empresa que fornece metadados de música, vídeo e esportes e tecnologias de reconhecimento automático de conteúdo para empresas e serviços de entretenimento em todo o mundo.

Segundo o site da companhia\footnote{http://www.gracenote.com/music/music-recognition/}, tradução nossa:

\begin{citacao}
O Gracenote MusicID\textregistered é o padrão para reconhecimento de música. Ele ajuda os fãs a desbloquear seus álbuns e faixas favoritos na nuvem e a descobrir novas músicas com seus celulares, além de permitir o monitoramento de músicas para detentores de direitos e profissionais do setor \cite{musicid1998}.
\end{citacao}

O Gracenote MusicID\textregistered, faz o reconhecimento de músicas que são tocadas ao seu redor, combinado ao uso de \textit{fingerprints} e correspondência de texto para identificar arquivos de música digital a um banco de dados mundial de informações musicais. Uma vez reconhecidos, os arquivos são organizados por nome de faixa, nome do álbum e caminhos de pastas para garantir que as músicas e os álbuns certos sejam sempre correspondidos.

%SHAZAM%
\subsection{Shazam}
\textit{Shazam Entertainment Ltd.} foi fundado em 2000 com a idéia de prover um serviço que pudesse conectar as pessoas à musica, permitindo a identificação da música através dos \textit{Smartphones}.

A aplicação usa o microfone do \textit{Smartphone} ou do computador para capturar uma pequena amostra de música, e então, realiza a identificação da música em um grande banco de dados com mais de 12 bilhões de músicas e, além disso, deve ter um baixo número de erros, ao mesmo tempo que tem uma alta taxa de acertos.

Segundo o site da companhia \footnote{https://www.shazam.com/pt/company}:

\begin{citacao}
Shazam é uma aplicação móvel que reconhece música e conteúdos de TV à sua volta. É a melhor maneira de descobrir, explorar e compartilhar a música e os conteúdos de TV que você mais gosta. Levamos 10 anos para alcançar 1 bilhão de tags, 10 meses para chegar a 2 bilhões, 3 meses para ir de 10 a 12 bilhões... É uma aplicação fantástica, agora disponível nas lojas da Apple e Android. E estamos sempre à procura de novas maneiras de encantar os nossos usuários \cite{shazam2000}.
\end{citacao}

Para o trecho de música capturado pela aplicação é criado uma \textit{fingerprint} ou impressão digital, tradução literal icorporada a palavra. Desta forma, é comparada com todas as outras \textit{fingerprints} derivadas das músicas no banco de dados. Se houver uma correspondência, é enviado informações da música para o usuário, como artista, álbum e título da música.

%SOUNDHOUND%
\subsection{SoundHound}
\textit{SoundHound Inc.} foi fundada em 2005, é uma empresa pioneira em desenvolvimento de aplicações para reconhecimento de voz, compreensão da linguagem natural, reconhecimento de som e tecnologias de busca.

Segundo o site da companhia\footnote{https://soundhound.com/about}, tradução nossa:

\begin{citacao}
Acreditamos em permitir que humanos interajam com as coisas ao seu redor da mesma forma como interagimos entre nós: falando naturalmente com telefones celulares, carros, TVs, caixas de música, máquinas de café, e todas as outras partes emergentes do mundo "conectado". Nosso produto mais recente, Hound, utiliza a nossa tecnologia \textit{Speech-to-Meaning}\texttrademark\ para mostrar uma experiência inovadora com os \textit{Smartphones}. Nosso produto \textit{SoundHound} aplica nossa tecnologia a música, permitindo as pessoas descobrir, explorar e compartilhar música ao seu redor, e até mesmo encontrar o nome daquela música presa em suas cabeças cantando ou cantarolando. E através da plataforma Houndify, capacitamos os desenvolvedores para fazerem parte dessa revolução \textit{speech-to-meaning} \cite{soundhound2005}.
\end{citacao}

É através da plataforma independente de Inteligência Artificial Houndify, combinado ao \textit{Automatic Speech Recognition} (ASR) e o \textit{Natural Language Understanding} (NLU) que permite a identificação das músicas de forma rápida e eficiente.

Desta forma, os dois produtos conhecidos no meio musical são:

\begin{enumerate}
    \item SoundHound Music Search \& Play\footnote{https://soundhound.com/soundhound}: É um aplicativo para \textit{Smartphones}, onde é possível descobrir, pesquisar e reproduzir qualquer música com controle de voz. Também é possível pesquisar músicas cantando ou cantarolando, tornando o único app no mundo que pode dar resultados, imediatamente e com precisão, ouvindo você cantar ou cantarolar.
    \item Midomi\footnote{https://www.midomi.com/}: Possui as mesmas características que o item anterior, porém possui versão para \textit{web} e a versão \textit{mobile} é destinado aos modelos mais antigos de \textit{Smartphones}.
\end{enumerate}

%DEEZER%
\subsection{Deezer}
A Deezer nasceu da necessidade de facilitar a vida de seu fundador para ouvir e fazer o \textit{download} de músicas. Com isso, o idealizador da plataforma desenvolveu o \textit{Blogmusik.net} em 2006. Devido a popularidade, houve objeção de detentores de direitos autores, o que gerou o fechamento do site. Pouco tempo depois, um acordo histórico foi assinado e o antigo site voltou ao ar com o nome de Deezer.

Segundo o site da companhia\footnote{https://www.deezer.com/br/company}:

\begin{citacao}
Na Deezer você ouve toda e qualquer música, na hora que quiser. Explore mais de 53 milhões de faixas (e a contagem continua) e descubra artistas e músicas que você vai amar com a recomendação personalizada dos Editores Deezer. A Deezer está em todos os seus dispositivos, tanto on-line como off-line, sem limites de escuta. Música na ponta de seus dedos para todos os momentos do seu dia: amanhecer, ir ao trabalho, relaxar, viver a vida...é só dar play! \cite{deezer2006}
\end{citacao}

A Deezer também conta com uma série de aplicativos que complementam a experiência musical do usuário. O \textit{Stateeztics}, por exemplo, é um in-app exclusivo que traça o perfil musical do usuário e mostra suas estatísticas de consumo a partir do seu histórico sonoro. Outra aplicação disponível é o \textit{Edjing}, que oferece mixagem de músicas com diversas ferramentas de efeitos digitais, além de contar com uma interface bastante intuitiva. Já o usuário que está aprendendo a tocar instrumentos musicais pode contar com o \textit{Chordify}, que reconhece o som que está tocando na Deezer e faz a transcrição automática da harmonia em cifras.

%SPOTIFY%
\subsection{Spotify}
\textit{Spotify Ltd.} foi fundado em 2006, é um serviço de streaming de música, podcast e vídeo, além de ser o mais popular e usado do mundo. A plataforma fornece conteúdo protegido providos de restrição pela gestão de direitos digitais de gravadoras e empresas de mídia. O Spotify é um serviço \textit{freemium}: possui recursos gratuitos com propagandas ou limitações, enquanto recursos adicionais, como qualidade de transmissão aprimorada e \textit{downloads} de música, são oferecidos para assinaturas pagas.

Segundo o site da companhia\footnote{https://www.spotify.com/br/about-us/contact/}:

\begin{citacao}
Com o Spotify, é fácil encontrar a música certa para cada momento – no seu telefone, computador, tablet e outros. Existem milhões de faixas no Spotify. Não importa se você está malhando, em uma festa ou relaxando, a música certa está sempre em suas mãos. Escolha o que quer ouvir ou deixe o Spotify surpreendê-lo. Você também pode navegar pelas coleções de músicas de amigos, artistas e celebridades, ou criar uma estação de rádio e simplesmente aproveitar. Produza a trilha sonora de sua vida com o Spotify. Assine ou ouça de graça \cite{spotify2006}.
\end{citacao}

A plataforma emprega um modelo de distribuição de dados híbrido com uma combinação de compartilhamento de dados ponto-a-ponto (P2P) e uma infraestrutura de servidor. 

O Spotify possui uma \textit{Web API} que permite que desenvolvedores integrem o conteúdo do Spotify em seus próprios aplicativos. O Spotify \textit{Web API} é um serviço com base na arquitetura REST, que retorna em formato JSON os dados sobre álbuns, artistas, faixas, playlists, entre outros. Para acessar outras informações é necessário autenticação OAuth.

%SOUNDCLOUD%
\subsection{SondCloud}
\textit{SoundCloud} foi criado em 2007, é uma plataforma on-line de publicação de áudio utilizada por profissionais de música. Nele os músicos podem colaborar, compartilhar, promover e distribuir suas composições.
Originalmente, o objetivo do site era permitir que profissionais da música trocassem ideias sobre as composições nas quais estão trabalhando, permitindo uma fácil colaboração e comunicação antes de um lançamento público. Hoje, o site também é utilizado por ouvintes e usuários da web em geral.

Segundo o site da companhia\footnote{https://soundcloud.com/pages/contact}, tradução nossa:

\begin{citacao}
Como a maior plataforma de música e áudio do mundo, o SoundCloud permite que as pessoas descubram e desfrutem da maior seleção de músicas da mais diversificada comunidade de criadores do mundo. Desde o seu lançamento em 2008, a plataforma tornou-se famosa por seu conteúdo e recursos exclusivos, incluindo a capacidade de compartilhar músicas e se conectar diretamente com artistas, bem como descobrir trilhas inovadoras, demonstrações brutas, podcasts e muito mais. Isso é possível graças a uma plataforma aberta que conecta diretamente os criadores e seus fãs em todo o mundo. Os criadores de música e áudio usam o SoundCloud para compartilhar e gerar receita com seu conteúdo com um público global, além de receber estatísticas detalhadas e feedback da comunidade do SoundCloud. Ainda não tem uma conta gratuita? \cite{soundcloud2007}
\end{citacao}

Os usuários registrados podem ouvir o máximo de conteúdo como quiserem e podem fazer o \textit{upload} de até 180 minutos de áudio ao seu perfil. Todos esses recursos são gratuitos e estão disponíveis para todos os usuários registrados do SoundCloud.

A plataforma possui uma API integrada a várias aplicações, que permitem fazer o \textit{upload} ou \textit{download} de música e arquivos de música.

O SoundCloud descreve as faixas de música graficamente como formas de onda e permite aos usuários comentar em partes específicas do áudio (conhecido como comentários cronometrados). Estes comentários são exibidos ao escutar a parte do áudio que estão se referindo. Outras característica inclui reposts, listas de reprodução, seguidores e \textit{downloads} digitais de cortesia.

%MUSIXMATCH%
\subsection{Musixmatch}
\textit{Musixmatch} foi criado em 2010, com o objetivo de mudar a forma como as pessoas experimentam música e letras.

Segundo o site da companhia\footnote{http://about.musixmatch.com/}, tradução nossa:

\begin{citacao}
Musixmatch é a maior plataforma de letras do mundo - onde você pode pesquisar, curtir e compartilhar letras de qualquer música, em qualquer lugar do mundo \cite{musixmatch2010}.
\end{citacao}

A plataforma pode ser acessada através do site e por aplicativo nos \textit{Smartphones}. O Musixmatch digitaliza todas as músicas da biblioteca de música do usuário e encontra letras para todas elas e apresenta a letra conforme a música que estiver sendo tocada. 

Possui também, a capacidade para capturar uma pequena amostra de música (mesma função encontrada em aplicativos como o Shazam e o Soundhound), além de identificar a letra da música e mantê-la sincronizada enquanto a música é tocada.

%ACRCLOUD%
\subsection{ACRCloud}
\textit{ACRCloud} foi criado em 2015, sendo o campeão no campeonato de \textit{Audio Fingerprinting} do MIREX2015, organizado pelo Laboratório Internacional de Avaliação de Sistemas de Recuperação de Informação Musical (IMIRSEL, sigla em inglês).

ACR (\textit{Automatic Content Recognition}) é uma tecnologia de identificação para reconhecimento de conteúdo reproduzido em um dispositivo de mídia. Permitindo que os usuários obtenham rapidamente informações detalhadas sobre o conteúdo que acabaram de experimentar sem qualquer entrada de texto ou esforços de pesquisa.

Segundo o site da companhia\footnote{https://www.acrcloud.com/docs/acrcloud/}, tradução nossa:

\begin{citacao}
O ACRCloud fornece serviços ACR em nuvem para ajudar empresas e desenvolvedores excelentes a criar aplicativos baseados em impressões digitais de áudio, como reconhecimento de áudio (suporte a música, vídeo, anúncios on-line e off-line), monitoramento de transmissão, interatividade de segunda tela, detecção de direitos autorais etc \cite{acrcloud2015}.
\end{citacao}

O ACRCloud é uma plataforma de microseriços na nuvem, que possui reconhecimento de música através de \textit{fingerprints}, além do monitoramento de transmissão com identificação e apresentação de conteúdo, entre outros.

Possui integração com serviços de música como o Spotify, Deezer, entre outros, que permite desenvolvedores acessarem diretamente esses serviços e oferecer links diretos para seus usuários. Também possui o reconhecimento de músicas através do canto.
%https://www.acrcloud.com/pt/music-recognition

%MUSIPEDIA%
\subsection{Musipedia}
%https://www.musipedia.org/about.html
Musipedia é uma enciclopédia aberta de música, criação inspirada no Wikipedia\footnote{https://www.wikipedia.org/}, para localização, edição e expansão de coleções de tons, melodias, e temas musicais.

Os conteúdos podem ser alteradas por qualquer usuário. Um conteúdo pode conter um pedaço de música, um arquivo MIDI, informações textuais sobre o trabalho e o compositor, e o código de Parson (uma descrição aproximada do contorno melódico).

Segundo o site da enciclopédia\footnote{https://www.musipedia.org/about.html}, tradução nossa:

\begin{citacao}
[...]Você pode tocá-lo em um teclado de piano, assobiar para o computador, simplesmente tocar o ritmo no teclado do computador ou usar o código Parsons \cite{musipedia}.
\end{citacao}

A enciclopédia utiliza o mecanismo de pesquisa de melodias \textit{Melodyhound}, onde é possível encontrar e identificar uma música, mesmo que a melodia seja tudo o que você saiba no momento.

%%ACADEMICO%%
\section{Acadêmico}

%MUSIC MINER%
\subsection{MusicMiner} \label{musicminer}
O \textit{Databionic MusicMiner}, desenvolvido como parte de um projeto de pesquisa do Grupo de Pesquisa em Databionics da Universidade de Marburg, na Alemanha, é um navegador para música baseado em técnicas de mineração de dados. Você pode criar \textit{MusicMaps} para visualizar a similaridade de músicas e artistas.

Segundo o site do projeto\footnote{http://musicminer.sourceforge.net/}, é composto pelas seguintes características:

\begin{itemize}
    \item Análise automática de uma árvore de pastas com arquivos de música (MP3, OGG, WMA, M4A, MP2, WAV);
    \item Descrição automática de arquivos de áudio digital por som;
    \item Criação de \textit{MusicMaps} para navegar pelo espaço sonoro com base no paradigma dos mapas geográficos;
    \item Criação visual de \textit{playlists};
    \item Pesquisa de similaridade na coleção de músicas com base no som;
    \item Navegação hierárquica personalizável da base de dados, por ex. gênero/artista/álbum ou ano/artista;
    \item Base de dados flexível, incluindo o armazenamento separado de vários artistas por música, álbuns e listas de reprodução como parte de uma lista de reprodução;
    \item Importação e exportação de meta informações baseadas em XML.
\end{itemize}

O \textit{MusicMiner} é escrito em Java para máxima portabilidade e publicado sob os termos da GPL (General Public License). Tendo como foco principal a pesquisa e o ensino.

Informações mais detalhadas sobre o projeto pode ser consultada em \cite{morchen2005} e \cite{musicminer}.

%YALE%
\subsection{YALE}
Um protótipo para a descoberta de conhecimento deve atender a vários requisitos:

\begin{enumerate}
    \item Deve ser flexível no que diz respeito aos métodos de pré-processamento e análise de dados, ou seja, deve oferecer uma gama muito ampla de métodos diferentes e que a incorporação de novos métodos seja fácil;
    \item Deve ser capaz de processar diferentes tipos de dados de entrada, como séries temporais ou dados de texto, sem esforço adicional para o usuário;
    \item Deve fornecer ampla funcionalidade para avaliação e otimização;
    \item Deve ser fácil de usar, ou seja, não deve exigir que o usuário aprenda um formalismo complexo para o uso.
    
\end{enumerate}

E o sistema Yale\footnote{http://yale.sf.net} foi desenvolvido para atender aos requisitos descritos acima. É um ambiente para experimentos de aprendizado de máquina e mineração de dados que apóiam o paradigma de prototipagem rápida. As aplicações da Yale abrangem tarefas de pesquisa e de mineração de dados do mundo real.

Atualmente, abrange o processamento de texto, áudio, séries temporais e multimídia, simulação de fluxo de dados e tratamento de desvio de conceito, armazenamento em cluster e mineração de dados distribuída.

Como curiosidade, o \textit{MusicMiner}, descrito na subseção \ref{musicminer} extrai os recursos de áudio necessários usando o plugin do sistema Yale.

Informações mais detalhadas sobre o projeto pode ser consultada em \cite{mierswa2006}

%CLAM%
\subsection{CLAM}
CLAM (\textit{C++ Library for Audio and Music})\footnote{http://clam-project.org/} é um \textit{framework}, desenvolvido em C++ no \textit{Music Technology Group} (MTG) na Universidade Pompeu Fabra em Barcelona, Espanha.

O \textit{framework} oferece uma plataforma completa de desenvolvimento e pesquisa para o domínio de áudio e música. Além de oferecer um modelo abstrato para sistemas de áudio, ele também inclui um repositório de algoritmos de processamento e tipos de dados, bem como diversas ferramentas, como entrada/saída de áudio ou MIDI.

O objetivo inicial do projeto CLAM era oferecer uma plataforma C++ de análise/síntese de som completa, flexível e independente de plataforma para atender às necessidades de todos os projetos do MTG. Esses objetivos iniciais mudaram um pouco desde então, principalmente porque a biblioteca não é mais vista como uma ferramenta interna para o MTG, mas como uma estrutura licenciada sob a GPL.

Informações mais detalhadas sobre o projeto pode ser consultada em \cite{amatriain2007}.

%MIR TOOLBOX%
\subsection{MIRtoolbox}
\textit{MIRtoolbox}\footnote{https://www.jyu.fi/hytk/fi/laitokset/mutku/en/research/materials/mirtoolbox} é uma caixa de ferramentas escritas em Matlab, dedicadas à extração de arquivos de áudio de recursos musicais como tonalidade, ritmo, estruturas, etc., desenvolvido pela Universidade de Jyväskylä.

Ele foi projetado especialmente com o objetivo de permitir o cálculo de uma grande variedade de recursos de bancos de dados de arquivos de áudio, que podem ser aplicados a análises estatísticas. Desta forma, poderia ser útil para a comunidade de pesquisa em Recuperação da Informação Musical (MIR), mas também para fins educacionais.

A caixa de ferramentas foi inicialmente concebida no contexto do projeto Brain Tuning financiado pela União Européia (FP6-NEST). Um dos principais objetivos foi investigar a relação entre características musicais e emoção induzida pela música e a atividade neural associada.

A \textit{MIRtoolbox} é oferecida gratuitamente à comunidade de pesquisa, e mais informações mais detalhadas sobre o projeto pode ser consultada em \cite{lartillot2007} e \cite{mirtoolbox}

%AMUSE%
\subsection{AMUSE}
AMUSE (\textit{Advanced Music Explorer})\footnote{https://sourceforge.net/projects/amuse-framework/} é um \textit{framework} licenciado sob a GPL e implementado em JAVA, desenvolvido pela TU Dortmund.

O \textit{framework} fornece diferentes funcionalidades, como: processamento de som convertendo arquivos de áudio MP3 em ondas sonoras; \textit{downsampling} para a conversão de arquivos de áudio mono; escalabilidade usando multi-threading e o gerenciamento eficiente do conjunto de dados que suporta diretamente o formato WEKA ARFF, além de um componente logger integrado. Oferecido gratuitamente à comunidade de pesquisa.

Informações mais detalhadas sobre o projeto pode ser consultada em \cite{vatolkin2010} e \cite{amuse}

%JAVA MIR%
\subsection{Java MIR}
jMIR (Java MIR)\footnote{http://jmir.sourceforge.net/} é um \textit{software} que compõem um conjunto de componentes desenvolvido na CIRMMT e Marianopolis College. Cada um dos componentes pode ser utilizado separadamente ou como um todo.

O software é de código livre implementado em Java para uso nas pesquisas de Recuperação de Informação Musical (MIR). Ele pode ser usado para estudar música na forma de gravações de áudio, codificações simbólicas e transcrições líricas, e também pode extrair informações culturais da Internet. Além de incluir ferramentas para gerenciar e criar perfis de grandes coleções de músicas e para verificar o áudio quanto a erros de produção. É bem documentado e inclui GUIs para aumentar a usabilidade geral.

O objetivo principal do \textit{software} é auxiliar nas pesquisas em classificação automática de música e a análise de similaridade, proporcionando as seguintes características:

\begin{itemize}
    \item Tornar tecnologias sofisticadas de reconhecimento de padrões acessíveis a pesquisadores de música com históricos técnicos e não técnicos;
    \item Eliminar duplicação redundante de esforço;
    \item Aumentar a cooperação e a comunicação entre os grupos de pesquisa;
    \begin{itemize}
        \item Facilitar o desenvolvimento iterativo e o compartilhamento de novas tecnologias MIR;
        \item Facilitar comparações objetivas de algoritmos.
    \end{itemize}
    \item Facilitar a pesquisa combinando características musicais de alto nível, baixo nível e culturais (ou seja, características simbólicas, áudio e web-minadas).
\end{itemize}

Informações mais detalhadas sobre o projeto estão disponíveis na publicações acadêmicas\footnote{http://jmir.sourceforge.net/publications.html}, assim como manuais e documentação para cada componente pode ser consultado em \cite{jmir} e \cite{mckay2010}.

%TUNEBOT%
\subsection{Tunebot}
%http://music.cs.northwestern.edu/data/tunebot/
\textit{Tunebot}\footnote{http://music.cs.northwestern.edu/data/tunebot/} é um projeto criado em 2015, com o objetivo de retornar ao usuário uma lista \textit{ranking} de músicas candidatas disponíveis no site do iTunes da Apple. O banco de dados do \textit{Tunebot} compara as músicas com as músicas cantadas pelos usuários, quanto mais os usuários contribuem com novas músicas, melhor é a eficiência do projeto no retorno dos resultados.

O projeto foi desenvolvido e é mantido por \textit{Interactive Audio Lab}\footnote{http://music.eecs.northwestern.edu/} na Universidade de Northwestern, nos Estados Unidos.

Outro objetivo do projeto é ajudar pesquisadores na área de reconhecimento de músicas que utilizam o algoritmo \textit{query by humming}, facilitando uma pesquisa mais precisa do desempenho do mundo real do que seria possível com conjuntos de dados existentes.

Informações mais detalhadas sobre o projeto pode ser consultada em \cite{pardo2012} e \cite{huq2010}.
