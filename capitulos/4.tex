\chapter{Análise e Resultados} \label{analiseResultados}
Este capítulo visa apresentar os seguintes itens: "Critérios de Análise", "Comparação e Análise"\ e "Resultados".

Para realização da análise proposta nesse trabalho, utilizaremos uma série de elementos que constituirão a metodologia de comparação entre as soluções que apresentem documentações públicas disponíveis e com informação necessária e suficiente.

\section{Critérios de Análise}

Com base nos conceitos apresentados por \citeonline[p.305]{wazlawick2012}, que define os atributos de qualidade internos, externos e de uso de produtos de software; e pela autora \citeonline{wangenheim2017}, que define heurísticas para assegurar que os produtos são usáveis, esta seção tem por objetivo apresentar os critérios que devem ser considerados na avaliação das soluções para busca de dados musicais, de modo a compará-las e facilitar a escolha pela mais apropriada para uma determinada situação. 

Assim sendo, os seguintes critérios e subcritérios são considerados:

\begin{itemize}
    \item Eficiência de desempenho: Trata-se da otimização do uso de recursos de tempo e espaço. Espera-se que o sistema seja o mais eficiente possível de acordo com o tipo de problema que ele soluciona:
    \begin{itemize}
        \item \textit{Comportamento em relação ao tempo}: Este critério mede o tempo que o sistema leva para processar suas funções, ou seja, o tempo de reconhecimento de uma música;
        \item \textit{Utilização de recursos}: Avalia a complexidade das estratégias e algoritmos utilizados na recuperação de informação musical;
        \item \textit{Bitrate}: Essa qualidade consiste no número médio de bits que será comprimido em um segundo de dados. A unidade utilizada é o KBPS ou 1000 BITS por segundo.
    \end{itemize}
    \item Adequação Funcional: A adequação funcional mede o grau em que o produto oferece funções que satisfazem necessidades estabelecidas e implicadas quando o produto é usado sob condições especificadas:
    \begin{itemize}
        \item \textit{Disponibilidade}: Avalia a disponibilidade da aplicação em diferentes plataformas;
        \item \textit{Modelo de desenvolvimento}: Avalia se a aplicação é de código aberto, dando a possibilidade para que qualquer um consulte, examine ou modifique o produto, e se permite extensões e/ou integrações com outras aplicações;
        \item \textit{Acessibilidade}: Avalia se a aplicação possui acesso ao acervo de músicas online e/ou offline;
        \item \textit{Busca de dados}: Avalia se a aplicação foi projetada para \textit{matching} exato ou aproximado;
        \item \textit{Inclusão da dados}: Avalia se a aplicação permite o envio de músicas feito pelo usuário;
        \item \textit{Modelo de pagamento}: Avalia o custo da aplicação, como por exemplo: Gratuito, Pago ou Freemium.
    \end{itemize}
    \item Usabilidade: Avalia o grau no qual o produto tem atributos que permitem que seja entendido, aprendido, usado e que seja atraente ao usuário, quando usado sob condições especificadas:
    \begin{itemize}
        \item \textit{Inteligibilidade}: Tem relação com o grau de facilidade que um usuário tem em entender os conceitos chave da aplicação e assim tornar-se competente no seu uso, como por exemplo: as funções mais utilizadas são facilmente acessadas?
        \item \textit{Operabilidade}: Avalia se a aplicação é fácil de usar e controlar, como por exemplo: Em campos onde há a necessidade de inserção de dados isso é evidente?
        \item \textit{Estética de interface com usuário}: Avalia se a interface com o usuário proporciona prazer e uma interação satisfatória, como por exemplo: O menu é esteticamente simples e claro?
        \item \textit{Proteção contra erro de usuário}: Avalia se o produto foi projetado para evitar que o usuário possa cometer erros, como por exemplo: Funções diferentes são apresentadas de maneira distinta ao usuário?
    \end{itemize}
\end{itemize}
